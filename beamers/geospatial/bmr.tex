% =========================================================================
\documentclass[show notes, aspectratio=1610]{beamer}
%\documentclass[aspectratio=1610]{beamer}

% ========================= Theme =========================================
\usetheme{Berkeley}
\usecolortheme{seahorse}

% ========================= Essential packages ============================
%\usepackage{hyperref}
%\hypersetup{
%    colorlinks = true,
%    linkcolor = blue,
%    citecolor = blue,
%    filecolor = blue,
%    urlcolor = blue
%}

% ========================= Frame notes systm ============================
%\usepackage{pgfpages}
%\setbeameroption{show notes on second screen}

% ========================= Plotting ======================================
\usepackage{calc}
\usepackage{tikz}
\usetikzlibrary{arrows,
                arrows.meta,
                calc,
		chains,
                quotes,
                positioning,
		shapes,
                shapes.geometric}
\usepackage{graphicx}
\usepackage{graphics}
\usepackage{pgfplots}
\pgfplotsset{width=7cm,compat=1.17}

%% ============================== Tabular =================================
\usepackage{booktabs}
\usepackage{tabularx,ragged2e}
\usepackage{array}
\usepackage{multirow}
\usepackage{siunitx}
  \sisetup{detect-all}
\usepackage{adjustbox}
\usepackage{rotating}
\usepackage{threeparttable}
\usepackage[justification=centering]{caption}
\usepackage{color, colortbl}

%% ============================== Text boxes ==============================
\usepackage[most]{tcolorbox}

% ========================= Infor on authors ==============================
%\title[Visualization Design]%
%{Visualization Design}
\title{Geospatial Data Visualization}
\author{S.~Santoni\inst{1}}
\institute{
	\inst{1}%
	Bayes Business School
	}
\date{MSc in Business Analytics, 2024/25}

% ============================ Colors =====================================
\definecolor{base_c}{rgb}{0.6,0,0}
\definecolor{comp_c}{rgb}{0.09803921568627451, 0.6901960784313725, 0.7529411764705882}
\definecolor{tri_1}{rgb}{0.09803921568627451, 0.7686274509803922, 0.19215686274509805}
\definecolor{tri_2}{rgb}{0.19215686274509805, 0.09803921568627451, 0.7686274509803922}

% ========================= TOC  ==========================================
\AtBeginSection[]
{
	\begin{frame}
		       \frametitle{Outline}
		       \tableofcontents[currentsection,currentsubsection]
	\end{frame}
}

% ========================= References ===================================
\usepackage[style=numeric,backend=biber]{biblatex}
\addbibresource{bibliography.bib}

% ========================= Document  ====================================
\begin{document}

\begin{frame}
	\titlepage
\end{frame}

\begin{frame}{Outline}
	\tableofcontents
\end{frame}

% ========================== Historical aspects ===========================
\section{The History of Geospatial Data Visualization}

\begin{frame}{Dr Snow's Study of Cholera Spread in 19\textsuperscript{th}
		Century London}{}

	\centering

	\includegraphics[width=0.75\textwidth]{images/SnowMap_Points.png}

	Go to the video: \url{https://www.youtube.com/watch?v=lNjrAXGRda4}
\end{frame}

\begin{frame}
	\frametitle{A Geospatial Visualization of Water Suppliers' Service Scope}
	\begin{figure}
		\begin{center}
			\includegraphics[width=0.6\textwidth]{images/water_supply.jpg}
		\end{center}
		\caption*{Source: Snow (1855)}\label{fig:water_supply}
	\end{figure}
\end{frame}

% ======================= Methodological aspects ===========================
\section{Methodological Aspects}

\begin{frame}{What Are the Methodological Roots of Geo-Spatial Analyses?}
	\begin{itemize}
		\item
		      Geographic Inforamtion Systems (GIS)
		\item
		      Remote sensoring
		\item
		      Elevation data
	\end{itemize}
\end{frame}

\begin{frame}{GIS}{}
	\begin{columns}
		\begin{column}{0.5\textwidth}
			\begin{itemize}
				\item
				      Computer mapping evolved with the computer itself in the 1960s
				\item
				      The term GIS began with the Canadian Department of Forestry
				      and Rural Development
				      \begin{itemize}
					      \item
					            Dr. Roger Tomlinson headed a team of 40
					            developers in an agreement with IBM to build the
					            Canada Geographic Information System (CGIS)
					      \item
					            The CGIS tracked the natural resources of Canada
					            and allowed the profiling of these features for further
					            analysis
					      \item
					            The CGIS stored each type of land cover as a different
					            layer
				      \end{itemize}
			\end{itemize}
		\end{column}
		\begin{column}{0.5\textwidth}
			\centering
			\includegraphics[width=0.8\textwidth]{images/canada_gis.jpeg}
		\end{column}
	\end{columns}
\end{frame}

\begin{frame}{An Example of GIS}{}
	\centering

	\includegraphics[width=0.5\textwidth]{images/example_of_gis.png}

	\small

	Source: Minard (early 1900s)
\end{frame}

\begin{frame}{Remote Sensoring}{}
	\begin{columns}
		\begin{column}{0.5\textwidth}
			\begin{itemize}
				\item
				      \textbf{Remote sensing} is the collection of information about an
				      object without making physical contact with that object
				\item
				      In the context of geospatial analysis, the object is usually
				      the Earth
				\item
				      By extracting features from images, modern remote sensing
				      technologies have enabled GIS
			\end{itemize}
		\end{column}
		\begin{column}{0.5\textwidth}
			\centering
			\includegraphics[width=0.8\textwidth]{images/56914f0d6cbe4.jpeg}
		\end{column}
	\end{columns}
\end{frame}

\begin{frame}
	\frametitle{Remote Sensoring in the Pre-Computer Vision Era}
	\framesubtitle{Leonardo Da Vinci's Map of the City of Imola (1502)}
	\begin{figure}
		\includegraphics[width=0.55\textwidth]{images/imola_map.jpg}
	\end{figure}

	\vspace{-3em}

	Got to the video: \url{https://www.youtube.com/watch?v=2gEwEcYnewE&t=8s}
\end{frame}

\begin{frame}{Elevation Data}{}
	\begin{columns}
		\begin{column}{0.5\textwidth}
			\begin{itemize}
				\item
				      A Digital Elevation Model (DEM) is a three-dimensional
				      representation of a planet's terrain (e.g., the Earth)
				\item
				      Before computers, representations of elevation data were
				      limited to topographic maps created through traditional land
				      surveys
				\item
				      Technology existed to create three-dimensional models from
				      stereoscopic images or physical models from materials such as
				      clay or wood
			\end{itemize}
		\end{column}
		\begin{column}{0.5\textwidth}
			\centering
			\includegraphics[width=0.8\textwidth]{images/worlddem_south_province_iceland_2012.jpg}
		\end{column}
	\end{columns}
\end{frame}

% ============== Geospatial data viz challenges ============================
\section{The Challenges of Geospatial Data Visualization}

\begin{frame}{Geographic Vs. Screen Coordinates}{}
	\begin{itemize}
		\item Geographic data are stored in a coordinate system representing a
		      grid overlaid on the Earth, which is \textcolor{blue}{three-dimensional}
		      and \textcolor{purple}{round}
		      \pause
		\item Screen coordinates, also known as pixel coordinates, represent a
		      grid of pixels on a \textcolor{blue}{two-dimensional} and \textcolor{purple}
		      {flat} computer screen
		      \pause
		\item Mapping $x$ and $y$ world coordinates to pixel coordinates is fairly
		      straightforward and involves a simple scaling algorithm
		      \pause
		\item However, if a $z$ coordinate exists, then a more complicated
		      transform must be performed to map coordinates from three-dimensional
		      space to a two-dimensional plane
	\end{itemize}
\end{frame}

\begin{frame}{Geographic Vs. Screen Coordinates}
	{How to build a stripped down version of a GIS}
	\begin{enumerate}
		\item Data model setup
		\item Fix map size
		\item Scaling the map
		\item Sample GIS function
		\item Render the map
	\end{enumerate}
\end{frame}

\begin{frame}{How to Build a Stripped Down Version of a GIS}
	{Step 1: data model setup}
	Let us set up the data for a sample USA state (Colorado) as a
	list with name, polygon points, and population

	\centering

	\includegraphics[width=0.6\textwidth]{images/colorado}
\end{frame}

\begin{frame}{How to Build a Stripped Down Version of a GIS}{Steps 2 - 5}
	\centering
	\Large
	See \texttt{tutorials/geospatial/gis.qmd}
\end{frame}

% ======================= Geospatial data viz modules ======================
\section{Python Modules for Geospatial Data Visualization}

\begin{frame}{The Python Geospatial Viz Ecosystem is VERY Rich!}{}
	\centering
	\Large
	Please refer to the Jupyter Notebook file \texttt{python\_modules.ipynb}
\end{frame}

% =========================== Bibliography =================================
%\begin{frame}
%	\frametitle{References}
%	\printbibliography
%\end{frame}

% =========================== Close ========================================
\end{document}
