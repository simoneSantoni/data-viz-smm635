% =========================================================================
\documentclass[notes, aspectratio=1610]{beamer}
%\documentclass[aspectratio=1610]{beamer}

% ========================= Theme =========================================
\usetheme{Berkeley}
\usecolortheme{seahorse}

% ========================= Essential packages ============================
%\usepackage{hyperref}
%\hypersetup{
%    colorlinks = true,
%    linkcolor = blue,
%    citecolor = blue,
%    filecolor = blue,
%    urlcolor = blue
%}

% ========================= Frame notes systm ============================
%\usepackage{pgfpages}
%\setbeameroption{show notes on second screen}

% ========================= Plotting ======================================
\usepackage{calc}
\usepackage{tikz}
\usetikzlibrary{arrows,
                arrows.meta,
                calc,
		chains,
                quotes,
                positioning,
		shapes,
                shapes.geometric}
\usepackage{graphicx}
\usepackage{graphics}
\usepackage{pgfplots}
\pgfplotsset{width=7cm,compat=1.17}

%% ============================== Tabular =================================
\usepackage{booktabs}
\usepackage{tabularx,ragged2e}
\usepackage{array}
\usepackage{multirow}
\usepackage{siunitx}
  \sisetup{detect-all}
\usepackage{adjustbox}
\usepackage{rotating}
\usepackage{threeparttable}
\usepackage[justification=centering]{caption}
\usepackage{color, colortbl}

%% ============================== Text boxes ==============================
\usepackage[most]{tcolorbox}		

% ========================= Infor on authors ==============================
\title[Visualization Design]%
{Visualization Design}
\subtitle{Graphical Perception and Colors}
\author{S.~Santoni\inst{1}\inst{2}}
\institute{
	\inst{1}%
	Bayes Business School
	\and
	\inst{2}%
	Soundcloud
	}
\date{MSc in Business Analytics, 2022/23}

% ============================ Colors =====================================
\definecolor{base_c}{rgb}{0.6,0,0}
\definecolor{comp_c}{rgb}{0.09803921568627451, 0.6901960784313725, 0.7529411764705882}
\definecolor{tri_1}{rgb}{0.09803921568627451, 0.7686274509803922, 0.19215686274509805}
\definecolor{tri_2}{rgb}{0.19215686274509805, 0.09803921568627451, 0.7686274509803922}

% ========================= TOC  ==========================================
\AtBeginSection[]
{
	\begin{frame}
		       \frametitle{Outline}
		       \tableofcontents[currentsection,currentsubsection]
	\end{frame}
}

% ========================= References ===================================
\usepackage[style=numeric,backend=biber]{biblatex}
\addbibresource{bibliography.bib}

% ========================= Document  ====================================
\begin{document}

\begin{frame}
	\titlepage
\end{frame}

\begin{frame}{Outline}
	\tableofcontents
\end{frame}

% ========================= Week 4 wrap up =================================
\section{Session \#4 Wrap Up}

\begin{frame}{Designing the `Lower-Level' Features of a Chart}
	{Source \cite[][page 61]{cairo2012}}
	\centering 
	\includegraphics[width=0.55\textwidth]{images/viz_wheel.jpeg}
\end{frame}

\begin{frame}{Designing the `Lower-Level' Features of a Chart}
	{Source \cite[][page 63]{cairo2012}}
	\centering 
	\includegraphics[width=0.9\textwidth]{images/viz_whell_comparison.jpeg}
\end{frame}

\begin{frame}{Chartjunk?}{}
	\centering 
	\includegraphics[width=0.55\textwidth]{images/chartjunk.png}
\end{frame}

\begin{frame}{Chartjunk or Memorable?}{}
	\centering 
	\includegraphics[width=0.8\textwidth]{images/chartjunk_vs_tufter.png}
\end{frame}

\begin{frame}{}{}
	\centering
	\includegraphics[width=0.7\textwidth]{images/memorable_data_viz.png}
\end{frame}

\begin{frame}{Empirical Evidence on Visualization Memorability}
	{The role of pictograms}
	\includegraphics[width=1\textwidth]{images/1_4}
\end{frame}

\begin{frame}{Empirical Evidence on Visualization Memorability}
	{The role of pictograms and color rating}
	\includegraphics[width=1\textwidth]{images/2_4}
\end{frame}

\begin{frame}{Empirical Evidence on Visualization Memorability}
	{The role of data-ink ratio and pictograms}
	\centering
	\includegraphics[width=0.6\textwidth]{images/3_4}
\end{frame}

\begin{frame}{Empirical Evidence on Visualization Memorability}
	{The role of pictograms and source}
	\centering
	\includegraphics[width=0.7\textwidth]{images/4_4}
\end{frame}

% ========================== Interactive dynamics and data viz =============
\section{Interactive Dynamics and Data Visualization}

\begin{frame}{From Static to Dynamic Visualization}{}
	\begin{center}
		\textbf{How to escape flatland?}
	\end{center}
	
	A single image typically provides answers to, at best, a handful of 
	questions

	\vspace{2em}

	\pause

\begin{columns}
	\begin{column}{0.5\textwidth}
		\begin{center}
			\textbf{Dynamic viz is my solution!}
		\end{center}

		Meaningful analysis consists of repeated explorations as users
		develop insights about significant relationships,
		domain-specific contextual influences, and causal patterns
	\end{column}

	\pause

	\begin{column}{0.5\textwidth}

		\begin{center}
			\textbf{Hold on...maybe not...}
		\end{center}
		Meaningful analysis consists of repeated explorations as
		users develop insights about significant relationships,
		domain-specific contextual influences, and causal
		patterns
	\end{column}
\end{columns}

\end{frame}

\begin{frame}{A Taxanomy of Interactive Dynamics for Visual Analysis}{}
	\footnotesize
	\centering
	\begin{table}
		\begin{tabular}{l|l}
			\hline
			\rowcolor{gray!25}
			Data \& View Specification
			%$\bullet$ Visualize data by choosing visual encodings \\
			%\rowcolor{gray!25}
			& $\bullet$ Filter out data to focus on relevant items\\
			\rowcolor{gray!25}
			& $\bullet$ Sort items to expose patterns\\
			\rowcolor{gray!25} 
			& $\bullet$ Derive values or models from source data \\
			\hline
			View Manipulation
			& $\bullet$ Select items to highlight, filter, or manipulate them \\
			& $\bullet$ Navigate to examine high-level patterns and low-level detail\\
			& $\bullet$ Coordinate views for linked, multi-dimensional exploration\\ 
			& $\bullet$ Organize multiple windows and workspaces\\
			\hline
			\rowcolor{gray!25}
			Process \& Provenance
			& $\bullet$ Record analysis histories for revisitation, review and sharing\\
			\rowcolor{gray!25}
			& $\bullet$ Annotate patterns to document findings\\
			\rowcolor{gray!25}
			& $\bullet$ Share views and annotations to enable collaboration\\
			\rowcolor{gray!25}
			& $\bullet$ Guide users through analysis tasks or stories\\
			\hline
		\end{tabular}
	\end{table}
\end{frame}

\begin{frame}{Let Us Get Things Done!}{}
	\centering
	\includegraphics[height=0.8\textheight]{images/dviz_overview}
\end{frame}

% =========================== Bibliography =================================
\begin{frame}
	\frametitle{References}
	\printbibliography
 \end{frame} 

\end{document}